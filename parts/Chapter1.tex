\section{Постановка задачи}
\label{sec:Chapter1} \index{Chapter1}

Работы по созданию эмбеддингов молекул можно разделить на два типа: использующие GNN и трансформеры. Оба подхода имеют ряд недостатков, заметных при использовании для решения прикладных химических задач. Например, при обучении почти никогда не используются физические свойства молекул, которые можно подавать в виде категориальных признаков. Большие модели, такие как BERT и многие типы GNN, достаточно плохо настраиваются под маленькие выборки веществ с специфическими физико-химическими свойствами. Кроме этого, на некоторых специфических молекулах трансформеры работают сильно хуже других (рис. \ref{fig:molecules} пункт б), а на молекулах со сложной структурой, с большим количеством атомов (рис. \ref{fig:molecules} пункт в) ошибаются графовые модели.
\newline
\textbf{Задачи исследования}:
\begin{enumerate}
    \setlength{\itemsep}{\smallskipamount}
    \item Параллельное обучение эмбеддингов: Используя BERT и GNN, обучить модели на молекулярных фингерпринтах и графовых представлениях с помощью известных подходов. Провалидировать полученные модели на задаче предсказывания физических свойств. Использовать дополнительную модель для управления выходами обеих моделей и применить dual-view consistency подход, который заключается в максимизации сходства между проекциями представлений молекул в проективных пространствах по различным метрикам.
    \item Файнтьюнинг моделей: Разработка алгоритма для тонкой настройки моделей под малые выборки с физическими свойствами, включая использование специализированного оптимизатора.
\end{enumerate}

\section*{Формальная постановка задачи}

\textbf{Дано:}
\begin{itemize}
  \item Множество молекул \( \mathcal{M} = \{m_1, m_2, ..., m_n\} \).
  \item Функции представления молекул в виде фингерпринтов \( f_{FP}: \mathcal{M} \rightarrow \mathbb{R}^d \) и графовых структур \( f_{G}: \mathcal{M} \rightarrow \mathbb{R}^d \).
  \item Набор физических свойств \( \mathcal{P} = \{p_1, p_2, ..., p_k\} \), соответствующих молекулам из \( \mathcal{M} \).
\end{itemize}

\textbf{Требуется:}
\begin{enumerate}
  \item Обучить модели BERT и GNN, \( \mathcal{B}: \mathbb{R}^d \rightarrow \mathbb{R}^h \) и \( \mathcal{G}: \mathbb{R}^d \rightarrow \mathbb{R}^h \), соответственно, используя известные подходы на представлениях молекул \( f_{FP}(m_i) \) и \( f_{G}(m_i) \).
  \item Провалидировать модели на задаче предсказания физических свойств \( \mathcal{P} \).
  \item Использовать дополнительную модель \( \mathcal{D} \) для управления выходами моделей BERT и GNN.
  \item Применить подход dual-view consistency, который формализуется как задача оптимизации:
  \[
  \max_{\mathcal{B}, \mathcal{G}} \sum_{i=1}^{n} \text{sim}\left( \mathcal{D}\left(\mathcal{B}\left(f_{FP}(m_i)\right)\right), \mathcal{D}\left(\mathcal{G}\left(f_{G}(m_i)\right)\right) \right),
  \]
  где \( \text{sim}(\cdot, \cdot) \) — метрика сходства, например, косинусное сходство.
\end{enumerate}




\newpage
