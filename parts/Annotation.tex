\stepcounter{page}
\begin{abstract}

    \begin{center}
        \large{Fine-Tuning для трансформеров молекулярных структур} \\
    \large\textit{Орлов Алексей Алексеевич} \\[1 cm]
    \end{center}
    Использование моделей машинного обучения открывает широкие перспективы для эффективного прогнозирования свойств химических соединений. Однако получение размеченных данных о молекулах может занимать много времени и требовать привлечения серьёзных ресурсов. В связи с этим, использование стандартного подхода обучения с учителем может быть затруднено, поэтому в хемоинформатике достаточно часто применяются подходы, основанные на обучении без учителя -- например, некоторые языковые (ChemBERTa \cite{ChemBERTa-2}, mol2vec \cite{mol2vec}), графовые модели (MolCLR \cite{molclr}). Логичным продолжением является совмещение достоинств этих двух подходов, что и было проделано \cite{DMP}. В данной работе применяется аналогичный подход, используя ECFP \cite{ECFP} представление молекул и графовое представление. При работе с химическими соединениями люди часто сталкиваются с ситуацией, когда необходимо дообучение тяжёлой предобученной модели на небольшом узкоспециализированном датасете. Данная работа позволяет решить и эту задачу.

    \vfill

    

\end{abstract}
\newpage