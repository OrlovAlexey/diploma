\section{Заключение}
\label{sec:Chapter5} \index{Chapter5}

В ходе выполнения данной дипломной работы была успешно решена задача исследования и применения методов обучения и fine-tuning для моделей машинного обучения молекулярных структур. Были достигнуты следующие ключевые результаты:
\begin{enumerate}
    \item Исследование уже существующих подходов к классификации и регрессии свойств молекулярных структур. Выбор наиболее успешных подходов, для последующей их проверки и применения (реализации) в решении задачи.
    \item Обучение и fine-tuning существующих моделей. Проведено обучение, валидация и тестирование моделей RoBERTa, Graphormer и MolCLR с использованием различных подходов к инициализации и настройке параметров. 
    \item Разработка и обучение новой модели. Обьединение существующие моделей RoBERTa и MolCLR, их совместное обучение и валидация. Результаты показали высокую степень согласованности эмбеддингов, получаемых от обеих моделей, а также высокое качество обучения данных моделей.
    \item Исследование новых подходов к классификации и регрессии. В процессе работы были изучены и протестированы новые методы классификации и регрессии молекулярных свойств. Применение трансформеров и графовых нейронных сетей позволило значительно повысить точность предсказаний по сравнению с традиционными методами.
\end{enumerate}

Настоящая работа вносит вклад в развитие методов машинного обучения для анализа и предсказания свойств химических соединений, открывая новые возможности для их практического применения.
\newpage
