\section{Исследование и построение решения задачи}
\label{sec:Chapter3} \index{Chapter3}




\subsection{Декомпозиция задачи}
Для решения поставленной задачи необходимо разбить её на ряд подзадач, каждая из которых должна быть достаточно простой для непосредственного решения. Процесс декомпозиции включает в себя следующие шаги:

\begin{enumerate}
\item \textbf{Сбор требований и анализ проблемы:}
\begin{itemize}
    \item Определение целей исследования и задач, которые необходимо решить.
    \item Исследование существующих решений и определение их ограничений.
\end{itemize}

\item \textbf{Исследование и выбор подходов:}
\begin{itemize}
    \item Анализ различных методов и технологий, применимых для решения задачи.
    \item Оценка плюсов и минусов различных моделей и подходов на основе научных публикаций и предыдуших исследований.
    \item Выбор оптимального подхода для каждой из подзадач.
\end{itemize}

\item \textbf{Проектирование архитектуры системы:}
\begin{itemize}
    \item Разработка общей архитектуры решения, включающей интеграцию всех компонентов.
    \item Определение взаимодействий между различными частями системы (например, как данные будут передаваться между моделями RoBERTa и GNN).
\end{itemize}

\item \textbf{Подготовка данных:}
\begin{itemize}
    \item Сбор исходных данных.
    \item Очистка данных от выбросов и слишком больших молекул.
    \item Нормализация данныхю
    \item Преобразование данных в необходимые форматы (ECFP и графовое представление).
    \item Предобработка данных для Graphormer (добавление необходимых полей, обьединение графов в батчи).
\end{itemize}

\item \textbf{Разработка и настройка моделей:}
\begin{itemize}
    \item Разработка и настройка токенайзера для модели RoBERTa.
    \item Определение и настройка архитектур для модели GNN.
    \item Настройка модели Graphormer.
    \item Подбор гиперпараметров для моделей.
    \item Выбор наиболее подходящей модели для задачи.
    \item Разработка baseline-модели, которая объединяет в себе два подхода.
\end{itemize}

\item \textbf{Обучение и тестирование моделей:}
\begin{itemize}
    \item Обучение модели RoBERTa с использованием Masked Language Modeling.
    \item Обучение моделей GNN: MolCLR и Graphormer.
    \item Тестирование обученных моделей на валидационном наборе данных.
\end{itemize}

\item \textbf{Интеграция и согласование эмбеддингов:}
\begin{itemize}
    \item Проекция эмбеддингов из различных моделей в общее пространство.
    \item Оптимизация сходства эмбеддингов с использованием метрик, таких как косинусное расстояние.
\end{itemize}

\item \textbf{Анализ и интерпретация результатов:}
\begin{itemize}
    \item Проведение экспериментов для оценки качества полученных эмбеддингов.
    \item Интерпретация результатов и формулирование выводов.
\end{itemize}

\end{enumerate}

\newpage
